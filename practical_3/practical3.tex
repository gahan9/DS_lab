%%%%%%%%%%%%%%%%%%%%%%%%%%%%%%%%%%%%%%%%%
% Journal Article
% Database System
% Practical 3: Implementation of B+ Tree
%
% Gahan M. Saraiya
% 18MCEC10
%
%%%%%%%%%%%%%%%%%%%%%%%%%%%%%%%%%%%%%%%%%
%----------------------------------------------------------------------------------------
%       PACKAGES AND OTHER DOCUMENT CONFIGURATIONS
%----------------------------------------------------------------------------------------
\documentclass[paper=letter, fontsize=12pt]{article}
\usepackage[english]{babel} % English language/hyphenation
\usepackage{amsmath,amsfonts,amsthm} % Math packages
\usepackage[utf8]{inputenc}
\usepackage{float}
\usepackage{lipsum} % Package to generate dummy text throughout this template
\usepackage{blindtext}
\usepackage{graphicx} 
\usepackage{caption}
\usepackage{subcaption}
\usepackage[sc]{mathpazo} % Use the Palatino font
\usepackage[T1]{fontenc} % Use 8-bit encoding that has 256 glyphs
\usepackage{bbding}  % to use custom itemize font
\linespread{1.05} % Line spacing - Palatino needs more space between lines
\usepackage{microtype} % Slightly tweak font spacing for aesthetics
\usepackage[hmarginratio=1:1,top=32mm,columnsep=20pt]{geometry} % Document margins
\usepackage{multicol} % Used for the two-column layout of the document
%\usepackage[hang, small,labelfont=bf,up,textfont=it,up]{caption} % Custom captions under/above floats in tables or figures
\usepackage{booktabs} % Horizontal rules in tables
\usepackage{float} % Required for tables and figures in the multi-column environment - they need to be placed in specific locations with the [H] (e.g. \begin{table}[H])
\usepackage{hyperref} % For hyperlinks in the PDF
\usepackage{lettrine} % The lettrine is the first enlarged letter at the beginning of the text
\usepackage{paralist} % Used for the compactitem environment which makes bullet points with less space between them
\usepackage{abstract} % Allows abstract customization
\renewcommand{\abstractnamefont}{\normalfont\bfseries} % Set the "Abstract" text to bold
\renewcommand{\abstracttextfont}{\normalfont\small\itshape} % Set the abstract itself to small italic text
\usepackage{titlesec} % Allows customization of titles

\renewcommand\thesection{\Roman{section}} % Roman numerals for the sections
\renewcommand\thesubsection{\Roman{subsection}} % Roman numerals for subsections

\titleformat{\section}[block]{\large\scshape\centering}{\thesection.}{1em}{} % Change the look of the section titles
\titleformat{\subsection}[block]{\large}{\thesubsection.}{1em}{} % Change the look of the section titles
\newcommand{\horrule}[1]{\rule{\linewidth}{#1}} % Create horizontal rule command with 1 argument of height
\usepackage{fancyhdr} % Headers and footers
\pagestyle{fancy} % All pages have headers and footers
\fancyhead{} % Blank out the default header
\fancyfoot{} % Blank out the default footer


%----------------------------------------------------------------------------------------
%----------------------------------------------------------------------------------------
%       CUSTOM HEADER TEXT
%----------------------------------------------------------------------------------------
\fancyhead[C]{Institute of Technology, Nirma University $\bullet$ September 2018} % Custom header text
%----------------------------------------------------------------------------------------

\fancyfoot[RO,LE]{\thepage} % Custom footer text
%----------------------------------------------------------------------------------------
%       PACKAGE for code highlight
%----------------------------------------------------------------------------------------
\usepackage[utf8]{inputenc}
\usepackage[english]{babel}

\usepackage{minted} % for highlighting code sytax
%----------------------------------------------------------------------------------------
%       TITLE SECTION
%----------------------------------------------------------------------------------------
\title{\vspace{-15mm}\fontsize{24pt}{10pt}\selectfont\textbf{
		\underline{Practical 3}\\Implementation of B+ Tree}} % Article title
\author{\large{\textsc{
		Gahan M. Saraiya, 18MCEC10 }}\\[2mm]
%\thanks{A thank you or further information}\\ % Your name
\normalsize \href{mailto:18mcec10@nirmauni.ac.in}{18mcec10@nirmauni.ac.in}\\[2mm] % Your email address
}
\date{}
\hypersetup{
	colorlinks=true,
	linkcolor=blue,
	filecolor=magenta,      
	urlcolor=cyan,
	pdfauthor={Gahan M. Saraiya},
	pdfcreator={Gahan M. Saraiya},
	pdfproducer={Gahan M. Saraiya},
}
%----------------------------------------------------------------------------------------

\begin{document}
\maketitle % Insert title
\thispagestyle{fancy} % All pages have headers and footers

\newcommand*\tick{\item[\Checkmark]}
\newcommand*\arrow{\item[$\Rightarrow$]}
\newcommand*\fail{\item[\XSolidBrush]}

\section{Introduction}
\paragraph{}
Aim of this practical is to implement algorithm of B+ tree.

Supported Operation
\begin{itemize}
	\item Insert single item
	\item Insertion in bulk
	\item Deletion
	\item Search
	\item Range Search
\end{itemize}

\section{Implementation}

\inputminted[frame=lines, breaklines, linenos]{python}{../btree_implementation/bPlusTree.py}

\subsection*{Output}
\begin{figure}[H]
	\includegraphics*[width=440px]{01.png}
\end{figure}

\begin{figure}[H]
	\includegraphics*[width=440px]{02.png}
\end{figure}

\begin{figure}[H]
	\includegraphics*[width=440px]{03.png}
\end{figure}

\begin{figure}[H]
	\includegraphics*[width=440px]{04.png}
\end{figure}



\section{Summary}
\begin{itemize}
	\tick all leaves at the same lowest level
	\tick all nodes at least half full (except root)
	\tick Supports Range Query
	\fail sequential search overhead may rise if large number of record in result of range query
\end{itemize}

\begin{table}[H]
	Let $f$ be the degree of tree and $n$ be the total number of data then
	\centering
	\caption*{Space Complexity}
	\begin{tabular}{r | c | c | c | c}
		& \textbf{Max \# pointers} & \textbf{Max \# keys} & \textbf{Min \# pointers} & \textbf{Min \# keys} \\
		\hline
		\hline
		\textbf{Non-leaf} & $f$ & $f - 1$ & $\lceil f/2 \rceil$ & $\lceil f/2 \rceil - 1$ \\
		\textbf{Root} & $f$ & $f - 1$ & 2 & 1 \\
		\textbf{Leaf} & $f$ & $f - 1$ & $\lfloor f/2 \rfloor$ & $\lfloor f/2 \rfloor$ \\
	\end{tabular}
\end{table}


- Number of disk accesses proportional to the height of the B+ tree.
which is the \textit{\textbf{worst-case height}} of a B+ tree is:

\begin{equation}
	h \propto \log_f\frac{n+1}{2} \approx O(log_fn)
\end{equation}

\begin{table}[H]
	Let $f$ be the degree of tree and $n$ be the total number of data then
	\centering
	\caption*{Time Complexity}
	\begin{tabular}{r | c | l }
		& \textbf{Time Complexity} & \textbf{Remarks} \\
		\hline
		\hline
		\textbf{height} & $O(\log_fn)$ & \\
		\textbf{Root} & $O(f\log_fn)$ & linear search inside each nodes \\
		\textbf{search} & $O(\log_2f\log_fn)$  & binary search inside each node \\
		\textbf{insert} & $O(\log_fn)$ & if splitting not require \\
		\textbf{insert} & $O(f\log_fn)$  & if splitting require \\
		\textbf{insert} &  $O(\log_fn)$  & if merge not require  \\
		\textbf{insert} &  $O(f\log_fn)$  & if merge require \\
	\end{tabular}
\end{table}

%\subsection{Bubble Sort}
%\begin{figure}[H]
%	\centering
%	\includegraphics[scale=0.75]{../analysis/Bubble_sort_(Iterative).png}
%	\caption{Bubble Sort}
%\end{figure}
%
%\section*{Output 2}
%\inputminted[frame=lines, breaklines]{text}{output2.txt}
%
%\section*{Output 3}
%\inputminted[frame=lines, breaklines]{text}{output3.txt}

%\section{Summary}
%\begin{itemize}
%	\tick If directory fits in memory then point query requires only $ 1 $ disk access
%	\tick Empty buckets can be merge with it's split image when directory becomes half of size
%\end{itemize}
%----------------------------------------------------------------------------------------
%\end{multicols}
\end{document}
