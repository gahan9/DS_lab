%%%%%%%%%%%%%%%%%%%%%%%%%%%%%%%%%%%%%%%%%
% Journal Article
% Database System
% Practical 4: Extendsible Hashing
%
% Gahan M. Saraiya
% 18MCEC10
%
%%%%%%%%%%%%%%%%%%%%%%%%%%%%%%%%%%%%%%%%%
%----------------------------------------------------------------------------------------
%       PACKAGES AND OTHER DOCUMENT CONFIGURATIONS
%----------------------------------------------------------------------------------------
\documentclass[paper=letter, fontsize=12pt]{article}
\usepackage[english]{babel} % English language/hyphenation
\usepackage{amsmath,amsfonts,amsthm} % Math packages
\usepackage[utf8]{inputenc}
\usepackage{float}
\usepackage{lipsum} % Package to generate dummy text throughout this template
\usepackage{blindtext}
\usepackage{graphicx} 
\usepackage{caption}
\usepackage{subcaption}
\usepackage[sc]{mathpazo} % Use the Palatino font
\usepackage[T1]{fontenc} % Use 8-bit encoding that has 256 glyphs
\usepackage{bbding}  % to use custom itemize font
\linespread{1.05} % Line spacing - Palatino needs more space between lines
\usepackage{microtype} % Slightly tweak font spacing for aesthetics
\usepackage[hmarginratio=1:1,top=32mm,columnsep=20pt]{geometry} % Document margins
\usepackage{multicol} % Used for the two-column layout of the document
%\usepackage[hang, small,labelfont=bf,up,textfont=it,up]{caption} % Custom captions under/above floats in tables or figures
\usepackage{booktabs} % Horizontal rules in tables
\usepackage{float} % Required for tables and figures in the multi-column environment - they need to be placed in specific locations with the [H] (e.g. \begin{table}[H])
\usepackage{hyperref} % For hyperlinks in the PDF
\usepackage{lettrine} % The lettrine is the first enlarged letter at the beginning of the text
\usepackage{paralist} % Used for the compactitem environment which makes bullet points with less space between them
\usepackage{abstract} % Allows abstract customization
\renewcommand{\abstractnamefont}{\normalfont\bfseries} % Set the "Abstract" text to bold
\renewcommand{\abstracttextfont}{\normalfont\small\itshape} % Set the abstract itself to small italic text
\usepackage{titlesec} % Allows customization of titles

\renewcommand\thesection{\Roman{section}} % Roman numerals for the sections
\renewcommand\thesubsection{\Roman{subsection}} % Roman numerals for subsections

\titleformat{\section}[block]{\large\scshape\centering}{\thesection.}{1em}{} % Change the look of the section titles
\titleformat{\subsection}[block]{\large}{\thesubsection.}{1em}{} % Change the look of the section titles
\newcommand{\horrule}[1]{\rule{\linewidth}{#1}} % Create horizontal rule command with 1 argument of height
\usepackage{fancyhdr} % Headers and footers
\pagestyle{fancy} % All pages have headers and footers
\fancyhead{} % Blank out the default header
\fancyfoot{} % Blank out the default footer

\fancyhead[C]{Institute of Technology, Nirma University $\bullet$ September 2018} % Custom header text

\fancyfoot[RO,LE]{\thepage} % Custom footer text
%----------------------------------------------------------------------------------------
%       TITLE SECTION
%----------------------------------------------------------------------------------------
\title{\vspace{-15mm}\fontsize{24pt}{10pt}\selectfont\textbf{Practical 4: Extendsible Hashing}} % Article title
\author{
\large
{\textsc{Gahan Saraiya, 18MCEC10 }}\\[2mm]
%\thanks{A thank you or further information}\\ % Your name
\normalsize \href{mailto:18mcec10@nirmauni.ac.in}{18mcec10@nirmauni.ac.in}\\[2mm] % Your email address
}
\date{}
\hypersetup{
	colorlinks=true,
	linkcolor=blue,
	filecolor=magenta,      
	urlcolor=cyan,
	pdfauthor={Gahan Saraiya},
	pdfcreator={Gahan Saraiya},
	pdfproducer={Gahan Saraiya},
}
%----------------------------------------------------------------------------------------
\usepackage[utf8]{inputenc}
\usepackage[english]{babel}

\usepackage{minted} % for highlighting code sytax
\begin{document}
\maketitle % Insert title
\thispagestyle{fancy} % All pages have headers and footers

\newcommand*\tick{\item[\Checkmark]}
\newcommand*\arrow{\item[$\Rightarrow$]}
\newcommand*\fail{\item[\XSolidBrush]}

\section{Introduction}
\paragraph{}
Extendsible Hashing is a dynamic hash system which treats hash as bit string. It has hierarchical nature as it extends buffer as needed. Thus buckets are added dynamically.

\section{Logic}
\begin{itemize}
	\item Create Global bucket and initialize with single empty local bucket
	\item Add item in bucket
	\item To add item Least Significant bits (LSB) are considered to avoid overhead of recalculating padding MSB (most significant bit)
	\item until bucket overflows it can accommodate data
	\item When bucket is full global bucket capacity is to be increased by factor of two and overflowed local buckets are to be spitted
\end{itemize}

\section{Implementation}
The code is implemented in Python as below

\inputminted[frame=lines, breaklines, linenos]{python}{extendsibleHashing.py}

\section*{Output 1}
\inputminted[frame=lines, breaklines]{text}{output1.txt}

\section*{Output 2}
\inputminted[frame=lines, breaklines]{text}{output2.txt}

\section*{Output 3}
\inputminted[frame=lines, breaklines]{text}{output3.txt}

\section{Summary}
\begin{itemize}
	\tick If directory fits in memory then point query requires only $ 1 $ disk access
	\tick Empty buckets can be merge with it's split image when directory becomes half of size
	\fail need to maintain local and global depth
	\fail Even if directory size doubles only on demand/overflow in worst case it may be possible that directory is getting overflow with similar matching bits in which the data stored in it are utilized about fifty percent only as other spaces are empty we can conclude that it may cause wastage of space in worst case.
\end{itemize}

\setlength{\tabcolsep}{10pt} % Default value: 6pt
\renewcommand{\arraystretch}{1.5} % Default value: 1
\begin{table}[!ht]
\begin{flushleft}
\centering
\caption*{Time Complexity of Extendible Hashing for single record access}
\begin{tabular}{ l  c c }

\hline
{} &\multicolumn{2}{ c }{\textbf{Condition}}\\ 
\cline{1-3}
\textbf{} & \textbf{directory size < memory size} & \textbf{directory size > memory size} \\
\hline
Access & 1 & 2 \\ 
 \hline
\end{tabular}
\end{flushleft}
\end{table}


\subsection*{Space Complexity}
\begin{table}[H]
	\begin{tabular}{r l}
		R & Number of records \\
		B & Block Size \\
		N & Number of blocks \\
	\end{tabular}
\end{table}
\paragraph{Space Utilization}
\begin{equation}
	\frac{R}{B \times N}
\end{equation}
\paragraph{Average Utilization} $\ln{2} \approx 0.69$
%----------------------------------------------------------------------------------------
%\end{multicols}
\end{document}
