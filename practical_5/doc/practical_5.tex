%%%%%%%%%%%%%%%%%%%%%%%%%%%%%%%%%%%%%%%%%
% Journal Article
% Database System
% Practical 5: Linear Hashing
%
% Gahan M. Saraiya
% 18MCEC10
%
%%%%%%%%%%%%%%%%%%%%%%%%%%%%%%%%%%%%%%%%%
%----------------------------------------------------------------------------------------
%       PACKAGES AND OTHER DOCUMENT CONFIGURATIONS
%----------------------------------------------------------------------------------------
\documentclass[paper=letter, fontsize=12pt]{article}
\usepackage[english]{babel} % English language/hyphenation
\usepackage{amsmath,amsfonts,amsthm} % Math packages
\usepackage[utf8]{inputenc}
\usepackage{float}
\usepackage{lipsum} % Package to generate dummy text throughout this template
\usepackage{blindtext}
\usepackage{graphicx} 
\usepackage{caption}
\usepackage{subcaption}
\usepackage[sc]{mathpazo} % Use the Palatino font
\usepackage[T1]{fontenc} % Use 8-bit encoding that has 256 glyphs
\usepackage{bbding}  % to use custom itemize font
\linespread{1.05} % Line spacing - Palatino needs more space between lines
\usepackage{microtype} % Slightly tweak font spacing for aesthetics
\usepackage[hmarginratio=1:1,top=32mm,columnsep=20pt]{geometry} % Document margins
\usepackage{multicol} % Used for the two-column layout of the document
%\usepackage[hang, small,labelfont=bf,up,textfont=it,up]{caption} % Custom captions under/above floats in tables or figures
\usepackage{booktabs} % Horizontal rules in tables
\usepackage{float} % Required for tables and figures in the multi-column environment - they need to be placed in specific locations with the [H] (e.g. \begin{table}[H])
\usepackage{hyperref} % For hyperlinks in the PDF
\usepackage{lettrine} % The lettrine is the first enlarged letter at the beginning of the text
\usepackage{paralist} % Used for the compactitem environment which makes bullet points with less space between them
\usepackage{abstract} % Allows abstract customization
\renewcommand{\abstractnamefont}{\normalfont\bfseries} % Set the "Abstract" text to bold
\renewcommand{\abstracttextfont}{\normalfont\small\itshape} % Set the abstract itself to small italic text
\usepackage{titlesec} % Allows customization of titles

\renewcommand\thesection{\Roman{section}} % Roman numerals for the sections
\renewcommand\thesubsection{\Roman{subsection}} % Roman numerals for subsections

\titleformat{\section}[block]{\large\scshape\centering}{\thesection.}{1em}{} % Change the look of the section titles
\titleformat{\subsection}[block]{\large}{\thesubsection.}{1em}{} % Change the look of the section titles
\newcommand{\horrule}[1]{\rule{\linewidth}{#1}} % Create horizontal rule command with 1 argument of height
\usepackage{fancyhdr} % Headers and footers
\pagestyle{fancy} % All pages have headers and footers
\fancyhead{} % Blank out the default header
\fancyfoot{} % Blank out the default footer

\fancyhead[C]{Institute of Technology, Nirma University $\bullet$ October 2018} % Custom header text

\fancyfoot[RO,LE]{\thepage} % Custom footer text
%----------------------------------------------------------------------------------------
%       TITLE SECTION
%----------------------------------------------------------------------------------------
\title{\vspace{-15mm}\fontsize{24pt}{10pt}\selectfont\textbf{Practical 5: Linear Hashing}} % Article title
\author{
\large
{\textsc{Gahan Saraiya, 18MCEC10 }}\\[2mm]
%\thanks{A thank you or further information}\\ % Your name
\normalsize \href{mailto:18mcec10@nirmauni.ac.in}{18mcec10@nirmauni.ac.in}\\[2mm] % Your email address
}
\date{}
\hypersetup{
	colorlinks=true,
	linkcolor=blue,
	filecolor=magenta,      
	urlcolor=cyan,
	pdfauthor={Gahan Saraiya},
	pdfcreator={Gahan Saraiya},
	pdfproducer={Gahan Saraiya},
}
%----------------------------------------------------------------------------------------
\usepackage[utf8]{inputenc}
\usepackage[english]{babel}
\usepackage[utf8]{inputenc}
\usepackage{fourier} 
\usepackage{array}
\usepackage{makecell}

\renewcommand\theadalign{bc}
\renewcommand\theadfont{\bfseries}
\renewcommand\theadgape{\Gape[4pt]}
\renewcommand\cellgape{\Gape[4pt]}
\newcommand*\tick{\item[\Checkmark]}
\newcommand*\arrow{\item[$\Rightarrow$]}
\newcommand*\fail{\item[\XSolidBrush]}
\usepackage{minted} % for highlighting code sytax

\begin{document}
\maketitle % Insert title
\thispagestyle{fancy} % All pages have headers and footers


\section{Introduction}
\paragraph{} Aim of this practical is to implementing Linear Hashing.
	\begin{itemize}
		\item linear growth of bucket array
		\item dynamic decision to increase bucket size
		\item bucket split criteria - $average\ bucket\ occupancy > threshold$ 
	\end{itemize}

\section{Logic}
\begin{enumerate}
	\item Initialize 
		\begin{itemize}
			\item bucket array with 2 buckets initially
			\item phase to 1
			\item threshold to 70\%
			\item hash function $h(element, without\_splitting) = element\ \% \ 2^{phase\_number}$
			\item hash function $h(element, during\_splitting) = element\ \% \ 2^{phase\_number+1}$
		\end{itemize}
	\item For each Insertion check for Average occupancy and decide whether split require or not.
	\item can insert node directly into bucket if no splitting is require (when average occupancy < threshold) but might need chaining
	\item otherwise splitting to be done from index 0 (first bucket)
\end{enumerate}

\section{Implementation}
The code is implemented in Python as below

\inputminted[frame=lines, breaklines, linenos]{python}{../linear_hashing.py}

\subsubsection{Output}
\inputminted[frame=lines, breaklines]{text}{../output_cap2.txt}


\subsubsection{Output}
\inputminted[frame=lines, breaklines]{text}{../output_cap3.txt}

\subsubsection{Output}
\inputminted[frame=lines, breaklines]{text}{../output_cap10.txt}

\section{Summary}

\subsection*{\textbf{Comparison with Extendible Hashing}}
\begin{table}[H]
	\begin{tabular}{r l}
		$N$ & Number of records \\
		$B$ & Number of buckets\\
		$b$ & bucket capacity \\
		$s$ & Number of successful searches \\
		$u$ & Number of unsuccessful searches \\
		$b_{s}$ & 1 + number of buckets accessed for successful search \\
		$b_{u}$ & 1 + number of buckets accessed for unsuccessful search \\
	\end{tabular}
\end{table}

\begin{table}[H]
	\bgroup
	\def\arraystretch{2}%  1 is the default, change whatever you need
	\begin{tabular}{c | c | c}
		\thead{Factor} & \thead{Linear Hashing} & \thead{Extendible Hashing}\\
		\hline
		Storage utilization & \makecell{$\frac{N}{B \times b}$} & \makecell{$\frac{N}{B \times b}$} \\
		\makecell{Average unsuccessful \\search cost} & \makecell{$\frac{b_u}{u}$} & \makecell{$\frac{b_u}{u}$} \\
		\makecell{Average unsuccessful \\search cost} & \makecell{$\frac{b_s}{s}$} & \makecell{$\frac{b_s}{s}$} \\
		\makecell{Split(expansion) \\Cost} & \makecell{1 access to read primary buckets \\ + k accesses to read k overflow buckets \\ + 1 access to write old bucket \\ + extra accesses to write \\the overflow
		buckets \\attached to old and new buckets} & \makecell{1 access to write old bucket \\ + 1 access to write new bucket \\+ extra access to write the overflow buckets \\ attached to old and new buckets \\ + accesses needed to update \\the directory pointers if the directory \\resides on the secondary storage} \\
		Insertion Cost & Unsuccessful search cost + Split cost & Unsuccessful search cost + Split cost \\
	\end{tabular}
	\egroup
\end{table}

\subsubsection*{Conclusion}
\begin{itemize}
	\item instead of exponential growth as in extendible hashing it's directory structure grows linearly
	\tick more efficient in terms of space utilization compared to extendible hashing
	\tick hash function calculated dynamically
	\fail for skew or non uniform data overflow chaining might be bottleneck in terms of search complexity as it may needs linear search for many records
	\fail overhead of computing threshold and splitting node if require on each \textbf{insert} operation
\end{itemize}

%\setlength{\tabcolsep}{10pt} % Default value: 6pt
%\renewcommand{\arraystretch}{1.5} % Default value: 1
%\begin{table}[!ht]
%\begin{flushleft}
%\centering
%\caption*{Time Complexity of Extendible Hashing for single record access}
%\begin{tabular}{ l  c c }
%
%\hline
%{} &\multicolumn{2}{ c }{\textbf{Condition}}\\ 
%\cline{1-3}
%\textbf{} & \textbf{directory size < memory size} & \textbf{directory size > memory size} \\
%\hline
%Access & 1 & 2 \\ 
% \hline
%\end{tabular}
%\end{flushleft}
%\end{table}
%
%
%\subsection*{Space Complexity}
%\begin{table}[H]
%	\begin{tabular}{r l}
%		R & Number of records \\
%		B & Block Size \\
%		N & Number of blocks \\
%	\end{tabular}
%\end{table}
%\paragraph{Space Utilization}
%\begin{equation}
%	\frac{R}{B \times N}
%\end{equation}
%\paragraph{Average Utilization} $\ln{2} \approx 0.69$
%----------------------------------------------------------------------------------------
%\end{multicols}
\end{document}
